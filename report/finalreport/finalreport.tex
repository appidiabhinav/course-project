\documentclass[a4paper]{article}

\begin{document}

\title{Improving image classification with CNNs by exploiting selectivity in search and training data}

\author{Muhammad Iqbal Tawakal}

\maketitle

\begin{abstract}
This is abstract. To be fully inserted later...
\end{abstract}

\section{Introduction}

The advent of GPU computing and changed the landscape of this field . Starting with the seminal paper by Krizhevsky which shows. Neural network but then died down.

in all vision task such as and retrieval \cite{alicvpr2014}.



In the end, this independent course project focused on to further improve this result. By using, let's say 200-300 regions per images, it is hypothesized that the performance of the system will improve.

to be continued...

\section{Method}

\subsection{Convolutional Neural Network}
Main difference, the operation is local and there is shared weight between.
This in turn actually perform a convolution of a kernel to an image. There is a max layer.

The architecture, dubbed as AlexNet, and defeat the last state-of-the-art method by large margin.

\subsection{Selective Search}
Selective search region proposal algorithm works by combining the result of its base segmentation algorithm using different (after some duplicates have been removed) parameter such as, and color space.

\subsubsection{Choosing positive samples}
The positive samples are chosen from all region with certain threshold. The formula for .. is

\section{Experiment}
This would be the baseline method.
The experiment is performed.

\section{Result}
Table \ref{tab:baseline_ap} shows the baseline accuracy of the proposed method.

\section{Discussion}

As can be seen on figure , the accuracy especially improve on object which have arguably small size such as . However, there is decline in performance for large object

\section{Conclusion}
The method shows a small improvement in the terms of mAP over the baseline method.

\bibliographystyle{plain}
\bibliography{finalreport}

\end{document}

