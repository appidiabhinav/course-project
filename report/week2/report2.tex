\documentclass{article}

\usepackage{graphicx}
\usepackage{caption}
\usepackage{subcaption}

\author{Muhammad Iqbal Tawakal}
\title{Progress report 2}

\begin{document}

\maketitle

\section{Baseline Result}
This is the result of using CNN features for PASCAL VOC 2007 image classification. This result is computed by using features extracted from layer number 7 (the second fully-connected layer), 16 jittered images, and trained using SVM with slack variable $C$ set to 1.6. This result pretty much similar to the one reported on \cite{alicvpr2014}.

\begin{table}[h]\tiny
\centerline{    \begin{tabular}{c c c c c c c c c c c c c c c c c c c c c c}
        \hline
        & aero & bike & bird & boat & bottle & bus & car & cat & chair & cow & table & dog & horse & mbike & person & plant & sheep & sofa & train & tv & mAP \\
        \hline
        & 87.5 & 81.0 & 84.4 & 83.7 & 43.6 & 70.9 & 84.1 & 82.8 & 61.3 & 66.2 & 66.5 & 79.4 & 84.9 & 77.1 & 91.7 & 54.2 & 73.3 & 66.6 & 87.4 & 71.4 & 74.9\\
        \hline
    \end{tabular}
}
\caption{Average Precision}
\end{table}

\begin{figure}[H]
	\begin{subfigure}[h]{0.2\textwidth}
		\includegraphics[width=\textwidth]{../../plot/1_5.png}
	\end{subfigure}
	\begin{subfigure}[h]{0.2\textwidth}
		\includegraphics[width=\textwidth]{../../plot/2_5.png}
	\end{subfigure}
	\begin{subfigure}[h]{0.2\textwidth}
		\includegraphics[width=\textwidth]{../../plot/3_5.png}
	\end{subfigure}
	\begin{subfigure}[h]{0.2\textwidth}
		\includegraphics[width=\textwidth]{../../plot/4_5.png}
	\end{subfigure}
	\begin{subfigure}[h]{0.2\textwidth}
		\includegraphics[width=\textwidth]{../../plot/5_5.png}
	\end{subfigure}
	\begin{subfigure}[h]{0.2\textwidth}
		\includegraphics[width=\textwidth]{../../plot/6_5.png}
	\end{subfigure}
	\begin{subfigure}[h]{0.2\textwidth}
		\includegraphics[width=\textwidth]{../../plot/7_5.png}
	\end{subfigure}
	\begin{subfigure}[h]{0.2\textwidth}
		\includegraphics[width=\textwidth]{../../plot/8_5.png}
	\end{subfigure}
	\begin{subfigure}[h]{0.2\textwidth}
		\includegraphics[width=\textwidth]{../../plot/9_5.png}
	\end{subfigure}
	\begin{subfigure}[h]{0.2\textwidth}
		\includegraphics[width=\textwidth]{../../plot/10_5.png}
	\end{subfigure}
	\begin{subfigure}[h]{0.2\textwidth}
		\includegraphics[width=\textwidth]{../../plot/11_5.png}
	\end{subfigure}
	\begin{subfigure}[h]{0.2\textwidth}
		\includegraphics[width=\textwidth]{../../plot/12_5.png}
	\end{subfigure}
	\begin{subfigure}[h]{0.2\textwidth}
		\includegraphics[width=\textwidth]{../../plot/13_5.png}
	\end{subfigure}
	\begin{subfigure}[h]{0.2\textwidth}
		\includegraphics[width=\textwidth]{../../plot/14_5.png}
	\end{subfigure}
	\begin{subfigure}[h]{0.2\textwidth}
		\includegraphics[width=\textwidth]{../../plot/15_5.png}
	\end{subfigure}
	\begin{subfigure}[h]{0.2\textwidth}
		\includegraphics[width=\textwidth]{../../plot/16_5.png}
	\end{subfigure}
	\begin{subfigure}[h]{0.2\textwidth}
		\includegraphics[width=\textwidth]{../../plot/17_5.png}
	\end{subfigure}
	\begin{subfigure}[h]{0.2\textwidth}
		\includegraphics[width=\textwidth]{../../plot/18_5.png}
	\end{subfigure}
	\begin{subfigure}[h]{0.2\textwidth}
		\includegraphics[width=\textwidth]{../../plot/19_5.png}
	\end{subfigure}
	\begin{subfigure}[h]{0.2\textwidth}
		\includegraphics[width=\textwidth]{../../plot/20_5.png}
	\end{subfigure}
	\caption{Precision-Recall Curve for all classes}
\end{figure}

\section{with Selective Search}
The selective search is performed on every images. Each operation resulted in approximately 200-300 regions per image.
The parameter used for the original segmentation algorithm, threshold $k$ is set to 200. The color space used is HSV, and two similarity measures used are color and texture.
Higher number of proposal per images can be achieved by using the combining the similarity measures, color space, and threshold.

Implementation is still underway
\emph{to be continued...}

\bibliographystyle{plain}
\bibliography{report}

\end{document}

